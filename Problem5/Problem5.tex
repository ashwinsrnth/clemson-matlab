\documentclass{article}
\usepackage{graphicx}
\usepackage{minted}
\usepackage{parskip}
\usepackage{hyperref}

\definecolor{bg}{rgb}{0.95, 0.95, 0.95}

\begin{document}
\section*{Problem 5}

Matlab is great for \emph{post processing} of results from experiments
or simulations. Post processing is a means of representing results in
a way that \emph{reveals} something about data. In this problem, you
will post process the results of a heat transfer simulation (don't worry
about the physics, we did that bit for you).

We stored information about the problem and the results in text 
files \texttt{t.txt, X.txt, Y.txt and temps.txt}. \textbf{Review the relevant
slides to understand the problem and the text files}.

Your objective is to make an animation out of the results so that
we can understand how the heat transfer happened better.


\begin{minted}[bgcolor=bg]{octave}
                                           
    % This code reads the results of a heat conduction 
    % simulation
    % and makes an animation of them.
    %  
    % 
    %

    X = dlmread('X.txt');

    % Similarly, read the rest of the text files
    % Y.txt, t.txt and temps.txt
    % Save the contents to variables names 'Y', 't' 
    % and 'temps' respectively
    % You might want to try reading these on the
    % shell first, and checking their sizes.

    % ---------------- YOUR CODE HERE --------------- %



    % ----------------------------------------------- %

    % The next few lines of code read the first row
    % of 'temps', and reshape it to the same size
    % as 'X' and 'Y'. We then plot a 'contour' of the
    % temperatures. This image is the first 'frame' of
    % our movie.

    % Rewrite the code to:
    %
    % - Loop over all the time steps
    %       - Extract the temperatures 
    %         corresponding to the time step
    %       - Reshape the temperature vector extracted
    %       - Draw a contour plot of the temperatures
    %       - Add the plot to the 'list of frames'


    % --------------- REWRITE THIS CODE -------------- %
    temps = T(1,:);
    temps = reshape(temps, 40, 40);
    mcontourf(x, y, temps);
    F(1) = getframe;
    % ------------------------------------------------ %

    % Play the movie 5 times
    movie(F, 5);


\end{minted}
\end{document}