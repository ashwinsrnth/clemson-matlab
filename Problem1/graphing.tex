\documentclass{article}
\usepackage{graphicx}
\usepackage{minted}
\usepackage{parskip}
\usepackage{hyperref}

\definecolor{bg}{rgb}{0.95, 0.95, 0.95}
\begin{document}

\section*{Problem 1: }

You know how to spawn a vector of values and evaluate and graph a 
function. Let's do something cool with functions. Of course, some functions 
are more interesting than others. For instance, consider this one:

\begin{center}
$y = e^{-0.2t}sin(t)$
\end{center}

This function describes the behaviour of a damped spring with time.
See \url{http://en.wikipedia.org/wiki/File:Damped_spring.gif}. 

Notice that the paramter $t$ is a measure of \emph{time} in say, seconds. 
Graph $y$ for $t$ from 0 to 30 seconds. Convince yourself that this figure 
does indeed describe the behaviour of a damped spring.

Simply graphing a function can yield a surprising amount of information.
For instance, with the figure you generated, you should be able to:

\begin{itemize}
\item{Find the maximum amplitude (the first peak)}
\item{Find the time at which the 3rd peak is reached}
\item{Replace 0.2 with some other numbers
    between 0 and 1. Can you attach a physical significance
    to this number?}
\end{itemize}



\end{document}

