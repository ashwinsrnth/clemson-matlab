\documentclass{article}
\usepackage{graphicx}
\usepackage{minted}
\usepackage{parskip}
\usepackage{hyperref}
\usepackage{amsmath}
\definecolor{bg}{rgb}{0.95, 0.95, 0.95}

\begin{document}
\section*{Problem 3}

Frequently in science and engineering, we encounter \emph{update}
problems, where some variable is repeatedly updated over a period
of time. Think of a can of Orange Juice placed in a refrigerator. 
We can expect the temperature of the orange juice to \emph{change} 
over time, and after a long enough period, to attain the `ambient'
temperature inside the refrigerator. The following \emph{model} 
describes how the temperature of the orange juice changes with time.

\begin{align}
T_{i+1} = T_{i} - K dt (T_i - F)
\end{align}

Where:

$T_{i+1}$ : Temperature at the end of step $i$

$T_i$     : Temperature at the beginning of step $i$

$K$       : A constant parameter that depends on the thermal properties
            of the can of the orange juice

$dt$      : The length (or duration) of the step (in say, minutes)

$F$       : The ambient temperature inside the refrigerator, also the 
            \emph{final} temperature of the can of juice.


To perform the simulation, we move in `steps' of time. 
For each step, we \emph{update}
the value of the temperature based on the equation. We repeat
this process till we end up with a set of values that should look
something like this:

\begin{table}[h!]
\begin{tabular}{|l|l|}
\hline
Time (min) & Temperature ($^{o}$C) \\
\hline
0    & 25 \\
10   & 20 \\   
20   & 17 \\
30   & 15 \\
40   & 13 \\
50   & 12 \\
\hline 
\end{tabular}
\end{table}

The following program generates these values. Understand and complete
the code. You should only need to write one line:

\begin{minted}[bgcolor=bg]{octave}

    % Initial and ambient temperatures:
    T = 25;
    F = 10;
    % Simulation parameters:
    k = 0.05;
    dt= 0.2;

    t = 0;

    hold on
    for i = 1:100
        % Update the temperature according to the formula
        % ------------- Your code here -------------- %
        
        % ------------------------------------------- %
        t = t + dt; 
        plot(t, T, 'o')
    end
    hold off

\end{minted}

Experiment. Try different values for \texttt{T}, \texttt{F}
and the parameter \texttt{k}. Note that our simulation is
for 100 steps (check the loop). For different combinations
of parameters, estimate if we need more or less time steps.

\end{document}